\begin{flushright}
Gilleleie, August 1, 1835
\end{flushright}

As I have now tried to show in the preceding pages, this is
  how things actually looked to me.
But when I try to get clear about my life, everything looks
  different.
Just as it takes a long time for a child to learn to
  distinguish itself from objects and an equally long time to
  disengage itself from its surroundings, with the result that it
  stresses the objective side and says, for example, \say{me hit
  the horse,} so the same phenomenon is repeated in a higher
  spiritual sphere.
I therefore believed that I would possibly achieve more
  tranquility by taking another line of study, by directing my
  energies toward another goal.
I might have succeeded for a time in banishing a certain
  restlessness, but it probably would have come back more intense,
  like a fever after drinking cold water.

What I really need is to get clear about what I must do,
  \footnote{
  How often, when a person believes that he has the best grip
      on himself, it turns out that he has embraced a cloud instead
      of Juno.
  }
  not what I must know, except insofar as knowledge must precede
  every act.
What matters is to find a purpose, to see what it really is
  that God wills that \textit{I} shall do; the crucial thing is
  to find a truth which is truth \textit{for me},
  \footnote{
  Only then does one have an inner experience, but how many
      experience life's different impressions the way the sea sketches
      figures in the sand and then promptly erases them without a
      trace.
  }
  to find \textit{the idea for which I am willing to live and
  die.}
Of what use would it be to me to discover a so-called
  objective truth, to work thorugh the philosophical systems so
  that I could, if asked, make critical judgments about them,
  could point out the fallacies in each system; of what use would
  it be to me to be able to develop a theory of the state,
  getting details from the various sources and combining them into
  a whole, and constructing a world I did not live in but merely
  held up for others to see; of what use would it be to me to
  be able to formulate the meaning of Christianity, to be able to
  explain many specific points---if it had not deeper meaning
  \textit{for me and for my life?}
And the better I was at it, the more I saw others
  appropritate the creations of my mind, the more tragic my
  situation would be, not unlike the that of parents who in their
  poverty are forced to send their children out into the world
  and turn them over to the case of others.
Of what use would it be to me for turth to stand before me,
  cold and naked, not caring whether or not I acknowledged it,
  making me uneasy rather than trustingly receptive.
I certainly do not deny that I still accept an
  \textit{imperative of knowledge} and that through it men may be
  influenced, but \textit{then it must come alive in me}, and
  \textit{this} is what I now recognize as the most important of
  all.
This is what my soul thirsts for as the African deserts thirst
  for water.
This is what is lacking, and this is why I am like a man
  who has collected furniture, rented an apartment, but as yet as
  not found the beloved to share life's ups and downs with him.
But in order to find that idea---or, to put it more
  correctly---to find myself, it does no good to plunge still
  farther into the world.
That was just what I did before.
The reason I thought it would be good to throw myself into
  law was that I believed I could develop my keeness of mind in
  the many muddles and messes of life.
Here, too, was offered a whole mass of details in which I
  could lose myself; here, perhaps, with teh given facts, I could
  construct a totality, an organic view of criminal life, pursue
  it in all its dark aspects (here, too, a certain fraternity of
  spirit is very evident).
I also wanted to become a lawyer so that by putting myself in
  another's role I could, so to speak, find a substitute for my
  own life and by means of this external change find some
  diversion.
This is what I needed to lead a \textit{completely human life}
  and not merely one of \textit{knowledge,}
  \footnote{
  How close men, despite all their knowledge, usually live to
      madness?
  What is truth but to live for an idea?
  When all is said and done, everything is based on a
      postulate; but not until it no longer stands on the outside,
      not until one lives in it, does it cease to be a postulate.
  (Dialectic---Dispute)
  }
  so that I could base the development of my thought not
  on---yes, not on something called objective---something which in
  any case is not my own, but upon something which is bound up
  with the deepest roots of my existence, through which I am, so
  to speak, grafted into the divine, to which I cling fast even
  though the whole world may collapse.
\textit{This is what I need, and this is what I strive for.}
I find joy and refreshment in contemplating the great men who
  have found that precious stone for which they sell all, even
  their lives
  \footnote{
  Thus it will be easy for us the first time we recieve that
      ball of yarn from Ariadne (love) and then go through all the
      mazes of the labyrinth (life) and kill the monster.
  But how many there are who plunge into life (the labyrinth)
      without taking that precaution (the \textit{young} girls and the
      little boys who are sacrified every year to the Minotaurus---?
  }
  whether I see them becoming vigorously engaged in life,
  confidently proceeding on their chosen course without vacillating,
  or discover them off the beaten path, absorbed in themselves and
  in working toward their high goal.
I even honor and respect the by-path which lies toward their
  high goal.
It is this inward action of man, this God-side of man, which
  is decisive, not a mass of data, for the latter will no doubt
  follow and will not appear as accidental aggregates or as a
  succession of details, one after the other, without a system,
  without a focal point.
I, too, have certainly looked for this focal point.
I have vainly sought an anchor in the boundless sea of
  pleasure as well as in the depths of knowledge.
I have felt the almost irresistible power with which one
  pleasure reaches a hand to the next; I have also felt the
  boredom, the shattering, which follows on its heels.
I have tasted the fruits of the tree of knowledge and time
  and again have delighted in their savoriness.
But this joy was only in the moment of cognition and did not
  leave a deeper mark on me.
It seems to me that I have not drunk from the cup of wisdom
  but have fallen into it.
I have sought to find the principle for me life through
  resignation, by supposing that since everything proceeds according
  to inscrutable laws it could not be otherwise, by blunting my
  ambitions and the antennae of my vanity.
Because I could not get everything to suit me, I abdicated
  with a consciousness of my own competence, somewhat the way
  decrepit clergymen resign with pension.
What did I find?
Not my self, which is what I did seek to find in a way (I
  imaginged my soul, if I may say so, as shut up in a box with
  a sping lock, which external surroundings would release by
  pressing the spring).
---Consequently the seeking and finding of the Kingdom of Heaven
  was the first thing to be resolved.
But it is just as useless for a man to want first of all to
  decide the externals and after that the fundamentals as it is
  for a cosmic surface, to what bodies it should turn its light,
  to which its dark side, without first letter the harmony of
  centrifugal and centripetal forces realize its existence and
  letting the rest come of itself.
One must first learn to know himself before knowing anything
  else \selectlanguage{greek} (γνῶθι σεαθτόν). \selectlanguage{english}
Not until a man has inwardly understood \textit{himself} and
  then sees the course he is to take does his life gain peace
  and meaning; only then is he free of that irksome, sinister
  traveling companion ---that irony of life
  \footnote{
  It may very well in a certain sense remain, but he is able
      to bear the swalls of this life, for the more a man lives
      for an idea, the more easily he comes to sit on the \say{I
      wonder} seat before the whole world.
  ---Frequently, when one is most convinced that he understands
      himself, he is assaulted by the uneasy feeling that he has
      really only learned someone else's life by rote.
  }
  which manifests itself in the spehere of knowledge and invites
  true knowing to being with a not-knowing (Socrates),
  \footnote{
  There is also a proverb which says: \say{One hears the truth
      from children and the insane.}
  Here it is certainly not a question of having truth according
      to premises and conclusions, but how often have not the words
      of a child or an insane person thundered at the man who
      would not listen to an intellectual genius.
  }
  just as God created the world from nothing.
But in the waters of morality it is especially at home to
  those still have not entered the tradewinds of virtue.
Here it tumbles a person about in a horrible way, for a time
  lets him feel happy and content in his rsolve to go ahead
  along the right path, then hurls him into the abyss of despair.
Often it lulls a man to sleep with the thought, \say{After
  all, things cannot be otherwise,} only to awaken him suddenly to
  a rigorous interrogation.
Frequently it seems to let a veil of forgetfulness fall over
  the past, only to make every single trifle appear in a strong
  light again.
When he struggles along the right path, rejoincing in having
  overcome temptations power, there may come at almost hte same
  time, right on the heels of perfect victory, an apparently
  insignificant external circumstance which pusses him down, like
  Sisyphus, from the height of the crag.
Often when a person has concentrated on something, a minor
  external circumstance arises which destroys everything.
As in the case of a man who, weary of life, is about to
  throw himself into the Thames and at the crucial moment is
  halted by the sting of a mosquito.)
Frequently a person feels his very best when the illness is
  the worst, as in tuberculosis.
In vain he tries to resist it but he has not sufficient
  strength, and it is no help to him that he has gone through
  the same thing many times; the kind of practice acquired in
  this way does not apply here.
Just as no one who has been taught a great deal about
  swimming is able to keep afloat in a storm, but only the man
  who is intensely convinced and has experienced that he is
  actually lighter than water, so a person who lacks this inward
  point of poise is unable to keep afloat in life's storms.
---Only when a man has understood himself in this way is he
  able to maintain an independent existence and thus avoid
  surrendering his own \textit{I}.
How often we see (in a person when we extol that Greek
  historian because he knows how to appropriate an unfamiliar style
  so delusively like the original author's, instead of censuring
  him, since the first prize always goes to an author for having
  his own style---that is, a mode of expression and presentation
  qualified by his own individuality)---how often we see people who
  either out of mental-spiritual laziness live on the crumbs that
  fall from another's table or for more egotisitcal reasons seek
  to identify themselves with others, until eventually they believe
  it all, just like the liar through which frequent repetition of
  his stories.
Although I am still far from this kind of interior
  understanding of myself, with profound respect for its
  significance I have sought to preserve my
  individuality---worshipped the unknown God.
With a premature anxiety I have tried to avoid coming in close
  contact with those things whose force of attraction might be too
  powerful for me.
I have sought to appropriate much from them, studied their
  distinctive characteristics and meaning in human life, but at the
  same time guarded against coming, like the moth, too close to
  the flame.
I have had little to win or to lose in association with the
  ordinary run of men, partly because what they do---so-called
  practical life---does not interest me much, partly because their
  coldness and indifference to the spiritual and deeper currents in
  man alienate me even more from them.
With few exceptions my companions have had no special influence
  upon me.
A life that has not arrived at clarify about itself must
  necessarily exhibit an uneven side-surface; confronted by certain
  facts and their apparent disharmony, they simply halted there,
  for, as I see it, they did not have sufficient interset to
  seek a resolution in a higher harmony or to recognize the
  necessity of it.
Their opinion of me was always one-sided, and I have vacillated
  between putting too much or too little weight on what they said.
I have now withdrawn from their influence and the potential
  variations of my life's compass resulting from it.
Thus I am again standing at the point where I must begin
  again in another way.
I shall now calmly attempt to look at myself and begin to
  initiate inner action; for only thus will I be able, like a
  child calling itself \say{I} in its first consciously undertaken
  act, be able to call myself \say{I} in a profounder sense.

But that takes stamina, and it is not possible to harvest
  immediately what one has sown.
I will remember that philosopher's method of having his
  disciples keep silent for three years; then I dare say it will
  come.
Just as one does not begin a feast at sunrise but at sundown,
  just so in the spiritual world one must first work forward for
  some time before the sun really shines for us and rises in all
  its glory; for although it is true as it says that God lets
  his sun shine upon the good and the evil and lets the rain
  fall on the just and the unjust, it is not so in the
  spiritual world.
So let the die be cast---I am crossing the Rubicon!
No doubt this road takes me \textit{into battle}, but I will
  not renounce it.
I will not lament the past---why lament?
I will work energetically and not waste time in regrets, like
  the person stuck in a bog and first calculating how far he has
  sunk without recognizing that during the time he spends on that
  he is sinking still deeper.
I will hurry along the path I have found and shout to
  everyone I meet: Do not look back as Lot's wife did, but
  remember that we are struggling up a hill.

\begin{flushright}
1 A 75 August 1, 1835
\end{flushright}
