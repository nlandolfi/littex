\documentclass[9pt]{extarticle}


%%% simple slides macros.tex
%%% N. C. Landolfi; 15 December 2021
%%% Adapted from zslides etc. of S. Lall.

\usepackage[T1]{fontenc}
\usepackage[paperwidth=16cm,paperheight=9cm,left=4mm,right=8mm,top=1mm,bottom=5mm]{geometry}
\usepackage{amsbsy}
\usepackage{amscd}
\usepackage{amsfonts}
\usepackage{amsmath}
\usepackage{amsmath}
\usepackage{amsopn}
\usepackage{amssymb}
\usepackage{amstext}
\usepackage{amsthm}
\usepackage{array}
\usepackage{ccfonts}
\usepackage{enumitem}
\usepackage{environ}
\usepackage{graphicx}
\usepackage{hyperref}
\usepackage{latexsym}
\usepackage{lipsum}
\usepackage{mathrsfs}
\usepackage{mathtools}
\usepackage{overpic}
\usepackage{subcaption}
\usepackage{tabto}
\usepackage{upgreek}
\usepackage{xcolor}


\renewcommand{\itdefault}{sl}
\renewcommand{\rmdefault}{cmss}

\parindent 0pt
\parskip 5pt
\normalsize
\topskip 1.2\baselineskip

% \definecolor{blue365}{hsb}{0.6666,1,0.6}
% \definecolor{red365}{rgb}{0.6,0,0}
\definecolor{blue365}{RGB}{0,21,138}
\definecolor{red365}{RGB}{153,2,0}
%\definecolor{blue365}{RGB}{0,154,220}
%\definecolor{red365}{RGB}{154,202,60}

%%%%%%%%%%%%%%%%%%%%%%%%%%%%%%%%%%%%%%%%%%%%%%%%%%%%%%%%%%%%%%%%%%%%%%%%%%%%%%


\def\numline{\raise 9mm\hbox to \textwidth{\textcolor{black}\hfill
    \fontfamily{cmss}\fontshape{n}\fontsize{6pt}{10}\selectfont\thepage}}
\makeatletter
\def\ps@finalheadings{%
  \let\@oddhead\@empty
  \let\@evenhead\@oddhead
  \def\@oddfoot{\numline}%
  \let\@evenfoot\@oddfoot
  \let\@mkboth\@gobbletwo
  \let\sectionmark\@gobble
  \let\subsectionmark\@gobble
}
\makeatother

\pagestyle{finalheadings}
\thispagestyle{finalheadings}

% bullets
\def\tribullet{{\small$\blacktriangleright$}}
\def\bluebullet{\textcolor{blue365}{\tribullet}}

% make second level lists smaller
\setlist[itemize]{label=\bluebullet}
\setlist[itemize, 2]{topsep=2pt,itemsep=2pt,before=\small}


%% the syntax is {title}{authors}{affiliation}{author shortnames}{revision}
\newcommand{\titleslide}[5]{
  \clearpage
  \hbox to \textwidth{\fontfamily{cmss}\fontshape{n}\fontsize{6pt}{10}\selectfont
    \textcolor{gray}{#3 \hfill #4}}

  \vspace*{1cm}

  \begin{center}
    \textcolor{blue365}{\LARGE \bf #1}
  \end{center}
  \vspace{1.5cm}

  \begin{center}
    #2 \\
    #3
  \end{center}
  \vfill
  \hbox to \textwidth{\fontfamily{cmss}\fontshape{n}\fontsize{6pt}{10}\selectfont \textcolor{gray}{#5  \hfill }}
}

\newcommand{\slide}[2]{
  \clearpage
  \textcolor{blue365}{\large \bf #1}
  \vfill
  #2
  \vfill
}
\newcommand{\slidei}[2]{\slide{#1}{\begin{itemize}#2\end{itemize}}}

\newcommand{\slidesection}[1]{
  \addcontentsline{toc}{section}{\hspace{1cm} #1 \hspace{1cm}\vspace{1cm}}
  \newpage
  \null
  \vfill
  \begin{center} \textcolor{blue365}{\Huge #1} \end{center}
  \vfill
}

%%% leftover magic from zslides
\makeatletter
% bumper slide
\def\section#1{\newpage\null\vfill\centerline{\color{blue365}\Huge #1}%
  \addtocontents{toc}{\hbox to \textwidth{\hskip 20pt{#1}\hfill}\vspace*{5mm}}%
  \vfill}


% table of contents
\def\tableofcontents{\@starttoc{toc}}
\def\contentsname{}
\makeatother
%%% leftover magic from zslides

\DeclareMathAlphabet{\mathbfsf}{\encodingdefault}{\sfdefault}{bx}{n}

\newcommand{\mathword}[1]{\mathop{\mathbfsf {#1}}}
\def\mathword#1{\mathop{\textup{#1}}}
\renewcommand{\emph}[1]{\textit{\textcolor{red365}{#1}}}


%%%%%%%%%%%%%%%%%%%%%%%%%%%%%%%%%%%%%%%%%%%%%%%%%%%%%%%%%%%%%%%%%%%%%%%%%%%%%%

\newcommand{\codestyle}[1]{\texttt{#1}}
\providecommand{\iestyle}[1]{{\fontfamily{cmr}\fontshape{it}\selectfont #1}}
\newcommand{\ie}{\iestyle{i.e.}}
\newcommand{\eg}{\iestyle{e.g.}}
\newcommand{\cf}{\iestyle{cf.}}
\newcommand{\etc}{\iestyle{etc.}}

\newcommand{\ones}{\mathbf{1}}
\newcommand{\zeros}{\mathbf{0}}

\newcommand{\R}{\textbf{R}}
\let\reals\R

%\newcommand{\N}{\mathbb{N}}
%\newcommand{\C}{\mathbb{C}}
%\newcommand{\Z}{\mathbb{Z}}
%\newcommand{\Symm}{\mathbb{S}}   % symmetric matrices

\newcommand{\half}{\frac{1}{2}}
\newcommand{\sample}{\mathop{\codestyle{sample}}}

% words where the command form is capitalized
\newcommand{\Span}{\mathword{span}}
\newcommand{\Expect}{\mathword{E{}}}
\newcommand{\Prob}{\mathword{Prob}}
\newcommand{\prob}{\mathword{prob}}

\newcommand{\loss}{\ell}
\newcommand{\eloss}{\mathcal L}

% words whose command form differs from the printed form
\newcommand{\nullspace}{\mathword{null}}

% words with standard lowercase command forms
\newcommand{\zscore}{\mathword{zscore}}
\newcommand{\median}{\mathword{median}}
\newcommand{\trace}{\mathword{trace}}
\newcommand{\rank}{\mathword{rank}}
\newcommand{\diag}{\mathword{diag}}
\newcommand{\var}{\mathword{var{}}}
\newcommand{\argmax}{\mathop{\mathrm{argmax}}}
\newcommand{\argmin}{\mathop{\mathrm{argmin}}}
\newcommand{\sign}{\mathword{sign}}
\newcommand{\rms}{\mathword{rms}}
\newcommand{\std}{\mathword{std}}
\newcommand{\vol}{\mathword{vol}}
\newcommand{\round}{\mathword{round}}
\newcommand{\sphdist}{\mathword{sphdist}}
\newcommand{\cone}{\mathword{cone}}
\newcommand{\re}{\mathword{Re}}
\newcommand{\im}{\mathword{Im}}
\newcommand{\range}{\mathword{range}}
\newcommand{\dist}{\mathword{dist}}
\newcommand{\avg}{\mathword{avg}}

\renewcommand{\dim}{\mathword{dim}}
\renewcommand{\min}{\mathword{min}}
\renewcommand{\max}{\mathword{max}}
\renewcommand{\det}{\mathword{det}}
\renewcommand{\vec}{\mathword{vec}}


\newcommand\indep{\protect\mathpalette{\protect\independenT}{\perp}}
\def\independenT#1#2{\mathrel{\rlap{$#1#2\mathsurround0pt$}\mkern3mu{#1#2}}}

\newcommand{\dyad}[1]{  #1 {#1}^\tp }
\newcommand{\indicator}[1]{\operatorname{\mathbf{I}}\parens*{#1}}


\newcommand{\MATLAB}{\codestyle{MATLAB}}

%%%%%%%%%%%%%%%%%%%%%%%%%%%%%%%%%%%%%%%%%%%%%%%%%%%%%%%%%%%%%%%%%%%%%%%%%%%%%%


% symbol for laplace transform
\def\laplace{\mathscr{L}}
\def\degrees{\ensuremath{{}^\circ}}

\newcommand{\imaginaryunit}{\mathbf{i}}
\newcommand{\convolution}{\mathbin{\ast}}
\newcommand{\given}{\mid}
\newcommand{\composition}{\mathbin{\circ}}
\newcommand{\hadamard}{\mathbin{\circ}}

% transpose
\def\tp{\mathsf{T}}

% probabilistic independence
\DeclareMathOperator{\ind}{\perp\hskip-0.6em\perp}



%%%%%%%%%%%%%%%%%%%%%%%%%%%%%%%%%%%%%%%%%%%%%%%%%%%%%%%%%%%%%%%%%%%%%%%%%%%%%%
% delimiters: needs amsmath and mathtools

\let\bl\bigl
\let\bbl\Bigl
\let\bbbl\biggl
\let\bbbbl\Biggl
\let\br\bigr
\let\bbr\Bigr
\let\bbbr\biggr
\let\bbbbr\Biggr

\newcommand{\vertt}{\big\vert}
\newcommand{\verttt}{\Big\vert}
\newcommand{\vertttt}{\bigg\vert}
\newcommand{\verttttt}{\Bigg\vert}


% norm
\DeclarePairedDelimiter{\norm}{\lVert}{\rVert}
\newcommand{\normm}[1]{\bl\lVert{#1}\br\rVert}
\newcommand{\normmm}[1]{\bbl\lVert{#1}\bbr\rVert}
\newcommand{\normmmm}[1]{\bbbl\lVert{#1}\bbbr\rVert}
\newcommand{\normmmmm}[1]{\bbbbl\lVert{#1}\bbbbr\rVert}

% use \abs{x} for |x|
% use \abs* for autosizing delimiters
\DeclarePairedDelimiter{\abs}{\lvert}{\rvert}
\newcommand{\abss}[1]{\bl\lvert{#1}\br\rvert}
\newcommand{\absss}[1]{\bbl\lvert{#1}\bbr\rvert}
\newcommand{\abssss}[1]{\bbbl\lvert{#1}\bbbr\rvert}
\newcommand{\absssss}[1]{\bbbbl\lvert{#1}\bbbbr\rvert}


% \cprob{a}{b} gives prob(a|b)
% use \cprob* for autosizing delimiters
\DeclarePairedDelimiterXPP{\cprob}[2]{\Prob}{(}{)}{}{#1 \, \delimsize\vert \, #2}

% use \set{stuff} for { stuff }
% use \set* for autosizing delimiters
\DeclarePairedDelimiter{\set}{\{}{\}}

% use \Set{a}{b} for {a | b}
% use \Set* for autosizing delimiters
\DeclarePairedDelimiterX{\Set}[2]{\{}{\}}{#1 \nonscript\;\delimsize\vert\nonscript\; #2}

% use \parens*{stuff} for (stuff) with autosizing delimeters
\DeclarePairedDelimiter{\parens}{(}{)}

% starred version is autosizing
\DeclarePairedDelimiterX{\innerproduct}[2]{\langle}{\rangle}{#1,#2}
\DeclarePairedDelimiter{\ceiling}{\lceil}{\rceil}
\DeclarePairedDelimiter{\floor}{\lfloor}{\rfloor}

%%%%%%%%%%%%%%%%%%%%%%%%%%%%%%%%%%%%%%%%%%%%%%%%%%%%%%%%%%%%%%%%%%%%%%%%%%%%%%
% extra items



%%%%%%%%%%%%%%%%%%%%%%%%%%%%%%%%%%%%%%%%%%%%%%%%%%%%%%%%%%%%%%%%%%%%%%%%%%%%%%
% matrices: needs amsmath and mathtools

% \newcommand{\bmat}[1]{\begin{bmatrix}#1\end{bmatrix}}
\def\bmat#1{\left[\hskip 0.5\arraycolsep\begin{matrix}#1\end{matrix}\hskip 0.5\arraycolsep\right]}

\newcommand{\mat}[1]{\begin{matrix}#1\end{matrix}}
\newcommand{\pmat}[1]{\begin{pmatrix}#1\end{pmatrix}}

\setcounter{MaxMatrixCols}{20}

% allow ddots at any angle
% \dddots{1}{1} produces dots at 45 degrees
% \dddots{2}{1} produces dots at 30 degrees
% and so on
\def\dddots#1#2{%
  \mbox{%
    \unitlength 0.083333in%r
    \begin{picture}(#1,#2)(0.25,0)
      \unitlength 0.0416666in%
      \multiput(0,2)(#1,-#2){3}{.}
    \end{picture}%
    }%
  }



%%%%%%%%%%%%%%%%%%%%%%%%%%%%%%%%%%%%%%%%%%%%%%%%%%%%%%%%%%%%%%%%%%%%%%%%%%%%%%
% references, needs cleverref

\usepackage[nameinlink]{cleveref}

%\Crefname{equation}{}{}
%\Crefname{equation}{}{}
%\crefname{figure}{figure}{figures}
%\Crefname{figure}{Figure}{Figures}
%\crefname{enumi}{}{}
%\Crefname{enumi}{}{}
%\creflabelformat{enumi}{#2#1#3)}

%%%%%%%%%%%%%%%%%%%%%%%%%%%%%%%%%%%%%%%%%%%%%%%%%%%%%%%%%%%%%%%%%%%%%%%%%%%%%%
% non math

%\def\eqnbox#1{\centerline{\fboxsep 5mm\framebox{$\displaystyle #1$}}}
%\let\boxed\undefined
%\NewEnviron{boxed}{\par\bigskip\eqnbox{\BODY}}
%
%\long\def\coordput#1{%
%  \AtBeginShipoutNext{\AtBeginShipoutUpperLeft{%
%      \setlength{\unitlength}{0.00125\paperwidth}%
%          #1}}}
%
%\long\def\coordputforeground#1{%
%  \AtBeginShipoutNext{\AtBeginShipoutUpperLeftForeground{%
%      \setlength{\unitlength}{0.00125\paperwidth}%
%          #1}}}
%% center a box vertically
%\newbox\qqboxa
%\def\vcent#1{\setbox\qqboxa\hbox{#1}\raise -0.5\ht\qqboxa\hbox{#1}}
%
%% center horizontally; useful in overpic
%\def\clap#1{\hbox to 0pt{\hss #1 \hss}}
%
%% inverse of vfill
%\def\vfillneg{\vskip 0pt plus -1fill}


\usepackage{nicefrac}
\renewcommand{\t}[1]{\emph{#1}}
\newcommand{\pa}[1]{\text{pa}_{#1}}
\newcommand{\rtree}[2]{\overrightarrow{#1}(#2)}


\newcommand{\CS}{\mathcal{S}}
\newcommand{\CU}{\mathcal{U}}
\newcommand{\CV}{\mathcal{V}}
\newcommand{\CX}{\mathcal{X}}
\newcommand{\CA}{\mathcal{A}}
\newcommand{\num}[1]{\abs{#1}}
\newcommand{\say}[1]{``#1''}
\newcommand{\PM}{\mathbfsf{P}}
\def\bmat#1{\left[\hskip 0.5\arraycolsep\begin{matrix}#1\end{matrix}\hskip 0.5\arraycolsep\right]}
\newcommand{\E}{\mathbfsf{E}}
\newcommand{\N}{\mathbfsf{N}}
\newcommand{\cov}{\mathword{cov}}


\newcommand{\unused}[1]{
  }

\newlength{\qqlen}
\def\verttop#1{\setbox\qqboxa\hbox{#1}\raise -\ht\qqboxa\hbox{#1}}
\long\def\coordput#1{%
  \AtBeginShipoutNext{\AtBeginShipoutUpperLeft{%
      \setlength{\unitlength}{0.00125\paperwidth}%
          #1}}}
\long\def\sideright#1{\coordput{\put(780,-70){\hskip -9mm{\llap{\raise -9mm\hbox{\verttop{#1}}}}}}}
\long\def\topright#1{\coordput{\put(800,0){\hskip -9mm{\llap{\raise -9mm\hbox{\verttop{#1}}}}}}}

% use \parens*{stuff} for (stuff) with autosizing delimeters
% \DeclarePairedDelimiter{\parens}{(}{)}
\newcommand{\transpose}[1]{{#1}^\tp}


\newcommand{\todo}[1]{\textcolor{red}{ {#1} }}

%% GP specific
\newcommand{\best}{\text{best}}
\newcommand{\mean}{\text{mean}}
% \newcommand{\cov}{\mathword{cov}}
% \newcommand{\var}{\mathword{var}}

\DeclarePairedDelimiter\ceil{\lceil}{\rceil}
% \DeclarePairedDelimiter\floor{\lfloor}{\rfloor}

\renewcommand{\mod}{\mathword{mod}}
\newcommand{\Z}{\mathbfsf{Z}}

\newcommand{\NA}{\text{A}}
\newcommand{\NT}{\text{T}}
\newcommand{\NC}{\text{C}}
\renewcommand{\NG}{\text{G}}
\newcommand{\NU}{\text{U}}
\newcommand{\CN}{\mathcal{N}}
\newcommand{\CG}{\mathcal{G}}
\newcommand{\CSTART}{\triangleright}
%\newcommand{\CSTOP}{\square}
\newcommand{\CSTOP}{\diamond}

\usepackage{booktabs}
\usepackage{longtable}


\begin{document}

    \titleslide
    { Basic mathematical genomics }
    {1}
    {2}
    {3}
    {4}
  \slidei{ DNA }{
  \item deoxyribonucleic acid (DNA) is a molecule which we can
    represent as a string
  \item a \t{nucleotide} (or \t{base pair}) is one of adenine,
    thymine, cytosine, and guanine
    \begin{itemize}
  \item we abbreviate the word \say{nucleotide} with \say{nt}
  \item we represent each of the four nucleotides with letters
        A, T, C, and G
  \item ribonucleic acid (RNA) has the nucleotide uracil (U)
        instead of thymine (T)
    \end{itemize}
  \item a \t{nucleotide string} is a sequence in the set
    $\set{\NA, \NT, \NC, \NG}$; e.g,
    $\NA\NT\NC\NG\NA\NT\NC\NA\NT\NC$
    \begin{itemize}
  \item in the \say{double helix} structure of DNA, $\NA$ binds
        with $\NT$ and $\NC$ binds with $\NG$, forming
        \say{cross-bars}
  \item we call $\NA$ the \t{nucleotide complement} of $\NT$, and
        vice versa; same for $\NC$ and $\NG$
  \item as a result, we can represent the double helix DNA as
        a single nucleotide string
    \end{itemize}
}
  \slidei{ Proteins }{
  \item a protein is a molecule which we can represent as a
    string
  \item an \t{amino acid} (also \t{residue}) is one of
    \begin{center}
        \begin{tabular}{rl|rl|rl|rl}
        \toprule
        name & symbol &
        name & symbol &
        name & symbol &
        name & symbol \\
        \midrule
        alanine       & A &
        arginine      & R &
        asparagine    & N &
        aspartate     & D \\
        cysteine      & C &
        glutamine     & Q &
        glutamate     & E &
        glycine       & G \\
        histidine     & H &
        isoleucine    & I &
        leucine       & L &
        lysine        & K \\
        methionine    & M &
        phenylalanine & F &
        proline       & P &
        serine        & S \\
        threonine     & T &
        tryptophan    & W &
        tyrosine      & Y &
        valine        & V \\
        \bottomrule
        \end{tabular}
      \end{center}
  \item an \t{amino acid string} is a sequence in $\{$A, R, N,
    D, C, Q, E, G, H, I, L, K, M, F, P, S, T, W, Y,
    V$\}$
    \begin{itemize}
  \item we denote this set by $\CA$, a mnemonic for \say{amino}
  \item different amino acid strings correspond to different
        proteins
  \item as a result, we can represent a protein as a single
        amino acid string
    \end{itemize}
}
  \slidei{ Codons }{
  \item nucleotides have semantic meaning in \t{non-overlapping}
    sequences of three
  \item a \t{nucleotide codon} (or \t{trinucleotide sequence}) is
    a length 3 nucleotide string; e.g., $\NA\NT\NC$
    \begin{itemize}
  \item codons encode an element of $\CA$ (an amino acid) or a
        \say{stop} (which we denote by $\CSTOP$)
  \item we partition the set $\set{\NA, \NT, \NC, \NG}^3$ of
        $4^3 = 64$ codons into 61 \t{amino codons} and 3 \t{stop
        codons}
    \end{itemize}
  \item a nucleotide string is \t{codon-aligned} if its length is
    a multiple of three
    \begin{itemize}
  \item a codon-aligned nucleotide string can be interpreted as a
        sequence of codons
  \item we know the \t{codon decoding function}
        $f: \set{\NA,\NC,\NT,\NG}^3  \to \CA \cup \set{\CSTOP}$
        \begin{itemize}
      \item for example, $f(\text{GCT})= \text{A}$ where the r.h.s.
                    is the symbol for the amino alanine
        \end{itemize}
  \item $f$ is not injective since two distinct codons may map
        to the same amino (or to $\CSTOP$)
        \begin{itemize}
      \item we call two codons with the same image under $f$
                    \t{synonyms}
      \item for example, CAU and CAC are synonyms for histidine;
                    i.e., $f(\text{CAU}) = f(\text{CAC}) = \text{H}$
        \end{itemize}
    \end{itemize}
}
  \slidei{ Codon Table }{
  \item it is easier to tabulate $f^{-1}$ since its codomain is
    smaller than its domain
    \begin{center}
        \begin{tabular}{cl|cl}
        \toprule
        symbol & codons; i.e., $f^{-1}(\text{symbol})$ &
        symbol & codons \\
        \midrule
        A & GCT, GCC, GCA, GCG           &
        I & ATT, ATC, ATA                \\
        R & CGT, CGC, CGA, CGG, AGA, AGG &
        L & CTT, CTC, CTA, CTG, TTA, TTG \\
        N & AAT, AAC                     &
        K & AAA, AAG                     \\
        D & GAT, GAC                     &
        M & ATG                          \\
        C & TGT, TGC                     &
        F & TTT, TTC                     \\
        Q & CAA, CAG                     &
        P & CCT, CCC, CCA, CCG           \\
        E & GAA, GAG                     &
        S & TCT, TCC, TCA, TCG, AGT, AGC \\
        G & GGT, GGC, GGA, GGG           &
        T & ACT, ACC, ACA, ACG           \\
        H & CAT, CAC                     &
        W & TGG                          \\
        $\CSTOP$ & TAA, TGA, TAG         &
        Y & TAT, TAC                     \\
          &                              &
        V & GTT, GTC, GTA, GTG           \\
        %$\CSTOP$ & TAA, TGA, TAG & & \\ %$\CSTART$ & ATG & \\
        \bottomrule
        \end{tabular}
      \end{center}
    \begin{itemize}
  \item the domain of $f$ is $\set{\NA, \NT, \NC, \NG}^3$ and the
        codomain of $f$ is $\CA \cup \set{\CSTOP}$
  \item $f^{-1}(x)$ is the set of domain elements of $f$ (in
        this case, codons) which map to $x \in \CA \cup \set{\CSTOP}$
    \end{itemize}
}
  \slidei{ Nucleotide senses }{
  \item naturally, we can extend $f$ to codon-aligned nucleotide
    strings by defining $s = \bar{f}(x)$ by
    \[
        s_i = f(
          \underbrace{
            x_{3(i-1)+1 } x_{3(i-1)+2} x_{3(i-1)+3}
          }_{\text{codon } i \text{ of } x}
        )
      \]
    \begin{itemize}
  \item we call $s$ the sense of $x$; for example, the sense
        of ATTCTTAAA is
        \[
            \bar{f}(
                \underbrace{\underline{\text{ATT}}}_{\text{I}}
                \underbrace{\underline{\text{CTT}}}_{\text{L}}
                \underbrace{\underline{\text{AAA}}}_{\text{K}}
            ) = \text{ILK}
          \]
    \end{itemize}
  \item since $f$ is not one-to-one, neither is $\bar{f}$
    \begin{itemize}
  \item $x$ and $y$ are \t{sense-equivalent} if they have the
        same sense; i.e., $\bar{f}(x) = \bar{f}(y)$
  \item roughly speaking, $x$ and $y$ are sense-equivalent if
        they \say{spell out the same thing}
  \item e.g., CGTCGC and CGACGG are sense-equivalent because
        $
            \bar{f}(
                \underbrace{\text{CGT}}_{\text{R}}
                \underbrace{\text{CGC}}_{\text{R}}
            ) = \bar{f}(
                \underbrace{\text{CGA}}_{\text{R}}
                \underbrace{\text{CGG}}_{\text{R}}
            ) = \text{RR}$
        \begin{itemize}
      \item in this case, because CGT, CGC, CGA, CGG are
                    synonyms for arginine (R)
        \end{itemize}
    \end{itemize}
}
  \slidei{ Nucleotide substitutions }{
  \item a \t{(nucleotide) substitution} (or \t{point mutation}) to
    a length $m$ nucleotide string is a pair $(j,b)$
    \begin{itemize}
  \item the \t{index} $j$ is in $\set{1, \dots, m}$ and the
        \t{replacement} nucleotide $b$ is in
        $\set{\NA, \NT, \NC, \NG}$
  \item the \t{(j,b)-mutation} of $x$ is the nucleotide string
        $y$ defined by $y_j = b$ and $y_i = x_i$ for all
        $i \neq j$
        \begin{itemize}
      \item i.e., $y$ is the same as $x$ except at index $j$,
                    where it has nucleotide $b$
      \item e.g., the $(3, \NA)$-mutation of $\text{CG\underline{T}}$
                    is $\text{CG\underline{A}}$ (we swapped T in position 3
                    with A)
        \end{itemize}
    \end{itemize}
  \item we classify substitutions on codon-aligned nucleotide
    sequences by their effect on the sense
    \begin{itemize}
  \item a substitution is \t{synonymous (silent)} if it does not
        change the sense
        \begin{itemize}
      \item e.g. (3, C) on CG\underline{T} with result
                    CG\underline{C}, since
                    $
                f(\text{CG\underline{T}}) =
                f(\text{CG\underline{C}}) =
                \text{R}
              $
        \end{itemize}
  \item a substitution is \t{nonsynonymous} if it changes the
        sense
        \begin{itemize}
      \item a substitution is \t{missense} if an amino codon
                    became a different amino codon
                    \begin{itemize}
              \item the \t{missense variants} of a protein are all
                                                proteins which differ with it by one amino in
                                                position
                    \end{itemize}
      \item a substitution is \t{nonsense (readstop)} if an amino
                    codon became a stop codon
      \item a substitution is \t{nonstop (readthrough)} if a stop
                    codon became amino codon
        \end{itemize}
    \end{itemize}
}
  

\end{document}
