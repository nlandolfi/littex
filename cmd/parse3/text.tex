A pack of wolves, a bunch of grapes, or a flock of pigeons
  are all examples of sets of things.
The mathematical concept of a set can be used as the
  foundation for all known mathematics.
The purpose of this little book is to develop the basic
  properties of sets.
Incidentally, to avoid terminological monotony, we shall sometimes
  say \textit{collection} instead of \textit{set}.
The word \say{class} is sometimes also used in this context,
  but there is a slight danger in doing so.
The reason is that in some approaches to set theory \say{class}
  has a special technical meaning.
We shall have occasion to refer to this again a little later.
  \footnote{
  See the end of Section 3.
  }

One thing that the development will not include is a definition
  of sets.
The situation is analogous to the familiar axiomatic approach to
  elementary geometry.
That approach does not offer a definition of points and lines;
  instead it describes what it is that one can do with those
  objects.
The semi-axiomatic point of view adopted here assumes the reader
  has the ordinary, human, intuitive (and frequently erroneous)
  understanding of what sets are; the purpose of the exposition is
  to delineate some of the many things that one can correctly do
  with them.

Sets, as they are usually conceived, have \textit{elements} or
  \textit{members}.
An element of a set may be a wolf, a grape, or a pigeon.
It is important to know that a set itself may also be an
  element of some other set.
Mathematics is full of examples of sets of sets.
A line, for instance, is a set of points; the set of all
  lines in the plane is a natural example of a set of sets (of
  points).
What may be surprising is not so much that sets may occur as
  elements, but that for mathematical purposes no other elements
  need every be considered.
In this book, in particular, we shall study sets, and sets of
  sets, and similar towers of sometimes frightening height and
  complexity---and nothing else.
By way of examples we might occasionally speak of sets of
  cabbages, and kings, and the like, but such usage is always to
  be construed as an illuminating parable only, and not as part
  of the theory that is being developed.

The principal concept of set theory, the one that in completely
  axiomatic studies is the principle primitive (undefined) concept,
  is that of \textit{belonging}.
If x belongs to A ($x$ is an element of $A$, $x$ is
  contained in $A$), we shall write
  \[
  x \in A.
  \]
This version of the Greek letter epsilon is so often used to
  denote belonging that its use to denote anything else is almost
  prohibited.